\documentclass[letterpaper,20pt,titlepage]{article}


%random comment

\newcommand{\cred}[1]{{\color{red}#1}}
\newcommand{\cblue}[1]{{\color{blue}#1}}


\usepackage{graphicx}                                        
\usepackage{amssymb}                                         
\usepackage{amsmath}                                         
\usepackage{amsthm}                                          

\usepackage{alltt}                                           
\usepackage{float}
\usepackage{color}
\usepackage{url}



\def\name{Jake Smith}


\parindent = 0.0 in
\parskip = 0.2 in

\begin{document}
\begin{center}

Senior Design 2016
\bigbreak
Problem Statement
\bigbreak
By Alex Hoffer, Jake Smith and Chen Chongxian
\bigbreak
Abstract:
\newline
With the growing dependency of industry and research work in biochemistry on computing, it becomes increasingly important for students studying biochemistry to be well-versed in state-of-the-art techniques like machine learning, a practice that enables biochemists to do remarkable things like discover new correlations between DNA and protein sequences. Since many of these students are burgeoning researchers or future industry professionals themselves, knowledge of machine learning is a necessity. The only problem is that these students are trained in biochemistry, not computer science. Therefore, there is a need to teach students of biochemistry about machine learning in a simple, clear, and entertaining way. This project aims at providing such an instructional tool to the students of Biochemistry and Biophysics at Oregon State University. This instructional tool will be a module hosted online for these students to learn machine learning concepts through experimentation with datasets and training related to the NCAA Men’s Division I Basketball Tournament, also known as March Madness. 
\newpage
\end{center}





\begin{section}{Problem Definition}
Our client is a researcher of Biochemistry and Biophysics. Machine learning is a useful tool for biochemical research and industry alike, but a problem is that computing skills of that kind are not emphasized in most Biochemistry and Biophysics curriculum. What makes matters more complex is that grasping machine learning through its application in Biochemistry and Biophysics is a particularly difficult way of learning it. Therefore, students who study this subject need to have the opportunity to learn machine learning for their future jobs in this field, whether it be in industry or research. They need for this learning process to be reasonably clear because they don’t have a background in computer science and machine learning approaches can be esoteric and confusing. 
\end{section}


\begin{section}{Proposed Solution}
Our solution to this problem is to make a learning module which will allow students of Biochemistry and Biophysics to make machine learning models based on data categories they select to predict which teams will advance in the NCAA Men’s Division I Basketball Tournament. The NCAA Men’s Division I Basketball Tournament consists of 68 college basketball teams that clinch berths based on winning their respective conferences in the regular season or are selected by the NCAA committee. They are then assigned to one of four regions (South, East, West, Midwest) based on the location of the college. Teams are seeded based on their win percentage in the regular season. So, for example, a first seeded team will play the sixteenth seeded team in their region, a second seeded team will play the fifteenth seeded, and so on. Once a team wins each of their games within their region, they enter into the Final Four, where they play a corresponding region’s winningest team. Finally, the two teams that emerge from these games battle each other and the winner of the final game is deemed the victor of the tournament. It is our client’s belief that the process of teams battling their way through the bracket provides a compelling opportunity to learn machine learning. Machine learning can be implemented so that a person selects which statistics they believe are useful in predicting whether a team will win their next game. These statistics are widely available on websites such as the NCAA.com and can be compiled into massive sheets of data. As with all machine learning, the question becomes which statistics are useful in prediction and which are useless or perhaps even counterproductive. In this way, Biochemistry and Biophysics students can easily see this fact by choosing seemingly obscure statistics and clearly impactful statistics alike to train data with and see which categories are valuable to producing a good bracket and which aren’t. Our project will consist of a user-friendly server where students can choose which basketball statistics to train data with and which statistics to avoid. Their models will be useful particularly in March, when the tournament begins, and their efforts will be easier to visualize. This solution will provide an easy interface by which these students can learn basic machine learning concepts based on predictions which are binary in nature (i.e. a team can either win or lose any particular game).
\end{section}


\begin{section}{Performance Metrics}


Our project’s success will be evaluated based on whether there is a fully functional learning module hosted on a server. Functionality will be measured by the program’s ability to generate a user’s predicted bracket which accurately reflects the statistical categories the user chose to train data on. The client has allowed us to determine the appropriate breadth of data available for users to choose from, and has recommended certain websites to mine from. One aspect of our success, then, is having a large mixture of categories by which the user can choose from. This is necessary because the success of the instructional tool is to the extent by which the student can understand machine learning concepts, and this knowledge will be solidified in seeing how brackets change based on which sets of data they use for training. Failure will consist of a) the server does not permit users to select data categories b) the server does not generate a bracket c) the bracket generated by the server doesn’t reflect the statistical categories the user selected. One possibility for our presentation at the Engineering Expo is our own personal brackets generated by our module, preferably ones we have successfully been able to optimize through careful selection of statistical categories and frequent trial-and-error generation of brackets.
\bigbreak
\bigbreak
\bigbreak
\end{section}

\newpage
\begin{section}{Terms and Agreement}
\bigbreak
\bigbreak
\bigbreak

\end{section}

\end{document}
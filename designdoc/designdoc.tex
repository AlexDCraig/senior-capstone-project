%% bare_jrnl.tex
%% V1.4b
%% 2015/08/26
%% by Michael Shell
%% see http://www.michaelshell.org/
%% for current contact information.
%%
%% This is a skeleton file demonstrating the use of IEEEtran.cls
%% (requires IEEEtran.cls version 1.8b or later) with an IEEE
%% journal paper.
%%
%% Support sites:
%% http://www.michaelshell.org/tex/ieeetran/
%% http://www.ctan.org/pkg/ieeetran
%% and
%% http://www.ieee.org/

%%*************************************************************************
%% Legal Notice:
%% This code is offered as-is without any warranty either expressed or
%% implied; without even the implied warranty of MERCHANTABILITY or
%% FITNESS FOR A PARTICULAR PURPOSE! 
%% User assumes all risk.
%% In no event shall the IEEE or any contributor to this code be liable for
%% any damages or losses, including, but not limited to, incidental,
%% consequential, or any other damages, resulting from the use or misuse
%% of any information contained here.
%%
%% All comments are the opinions of their respective authors and are not
%% necessarily endorsed by the IEEE.
%%
%% This work is distributed under the LaTeX Project Public License (LPPL)
%% ( http://www.latex-project.org/ ) version 1.3, and may be freely used,
%% distributed and modified. A copy of the LPPL, version 1.3, is included
%% in the base LaTeX documentation of all distributions of LaTeX released
%% 2003/12/01 or later.
%% Retain all contribution notices and credits.
%% ** Modified files should be clearly indicated as such, including  **
%% ** renaming them and changing author support contact information. **
%%*************************************************************************


% *** Authors should verify (and, if needed, correct) their LaTeX system  ***
% *** with the testflow diagnostic prior to trusting their LaTeX platform ***
% *** with production work. The IEEE's font choices and paper sizes can   ***
% *** trigger bugs that do not appear when using other class files.       ***                          ***
% The testflow support page is at:
% http://www.michaelshell.org/tex/testflow/


% Please refer to your journal's instructions for other
% options that should be set.
\documentclass[journal,onecolumn]{IEEEtran}
%
% If IEEEtran.cls has not been installed into the LaTeX system files,
% manually specify the path to it like:
% \documentclass[journal]{../sty/IEEEtran}





% Some very useful LaTeX packages include:
% (uncomment the ones you want to load)


% *** MISC UTILITY PACKAGES ***
%
%\usepackage{ifpdf}
% Heiko Oberdiek's ifpdf.sty is very useful if you need conditional
% compilation based on whether the output is pdf or dvi.
% usage:
% \ifpdf
%   % pdf code
% \else
%   % dvi code
% \fi
% The latest version of ifpdf.sty can be obtained from:
% http://www.ctan.org/pkg/ifpdf
% Also, note that IEEEtran.cls V1.7 and later provides a builtin
% \ifCLASSINFOpdf conditional that works the same way.
% When switching from latex to pdflatex and vice-versa, the compiler may
% have to be run twice to clear warning/error messages.






% *** CITATION PACKAGES ***
%
%\usepackage{cite}
% cite.sty was written by Donald Arseneau
% V1.6 and later of IEEEtran pre-defines the format of the cite.sty package
% \cite{} output to follow that of the IEEE. Loading the cite package will
% result in citation numbers being automatically sorted and properly
% "compressed/ranged". e.g., [1], [9], [2], [7], [5], [6] without using
% cite.sty will become [1], [2], [5]--[7], [9] using cite.sty. cite.sty's
% \cite will automatically add leading space, if needed. Use cite.sty's
% noadjust option (cite.sty V3.8 and later) if you want to turn this off
% such as if a citation ever needs to be enclosed in parenthesis.
% cite.sty is already installed on most LaTeX systems. Be sure and use
% version 5.0 (2009-03-20) and later if using hyperref.sty.
% The latest version can be obtained at:
% http://www.ctan.org/pkg/cite
% The documentation is contained in the cite.sty file itself.






% *** GRAPHICS RELATED PACKAGES ***
%
\ifCLASSINFOpdf
  % \usepackage[pdftex]{graphicx}
  % declare the path(s) where your graphic files are
  % \graphicspath{{../pdf/}{../jpeg/}}
  % and their extensions so you won't have to specify these with
  % every instance of \includegraphics
  % \DeclareGraphicsExtensions{.pdf,.jpeg,.png}
\else
  % or other class option (dvipsone, dvipdf, if not using dvips). graphicx
  % will default to the driver specified in the system graphics.cfg if no
  % driver is specified.
  % \usepackage[dvips]{graphicx}
  % declare the path(s) where your graphic files are
  % \graphicspath{{../eps/}}
  % and their extensions so you won't have to specify these with
  % every instance of \includegraphics
  % \DeclareGraphicsExtensions{.eps}
\fi
% graphicx was written by David Carlisle and Sebastian Rahtz. It is
% required if you want graphics, photos, etc. graphicx.sty is already
% installed on most LaTeX systems. The latest version and documentation
% can be obtained at: 
% http://www.ctan.org/pkg/graphicx
% Another good source of documentation is "Using Imported Graphics in
% LaTeX2e" by Keith Reckdahl which can be found at:
% http://www.ctan.org/pkg/epslatex
%
% latex, and pdflatex in dvi mode, support graphics in encapsulated
% postscript (.eps) format. pdflatex in pdf mode supports graphics
% in .pdf, .jpeg, .png and .mps (metapost) formats. Users should ensure
% that all non-photo figures use a vector format (.eps, .pdf, .mps) and
% not a bitmapped formats (.jpeg, .png). The IEEE frowns on bitmapped formats
% which can result in "jaggedy"/blurry rendering of lines and letters as
% well as large increases in file sizes.
%
% You can find documentation about the pdfTeX application at:
% http://www.tug.org/applications/pdftex





% *** MATH PACKAGES ***
%
%\usepackage{amsmath}
% A popular package from the American Mathematical Society that provides
% many useful and powerful commands for dealing with mathematics.
%
% Note that the amsmath package sets \interdisplaylinepenalty to 10000
% thus preventing page breaks from occurring within multiline equations. Use:
%\interdisplaylinepenalty=2500
% after loading amsmath to restore such page breaks as IEEEtran.cls normally
% does. amsmath.sty is already installed on most LaTeX systems. The latest
% version and documentation can be obtained at:
% http://www.ctan.org/pkg/amsmath





% *** SPECIALIZED LIST PACKAGES ***
%
%\usepackage{algorithmic}
% algorithmic.sty was written by Peter Williams and Rogerio Brito.
% This package provides an algorithmic environment fo describing algorithms.
% You can use the algorithmic environment in-text or within a figure
% environment to provide for a floating algorithm. Do NOT use the algorithm
% floating environment provided by algorithm.sty (by the same authors) or
% algorithm2e.sty (by Christophe Fiorio) as the IEEE does not use dedicated
% algorithm float types and packages that provide these will not provide
% correct IEEE style captions. The latest version and documentation of
% algorithmic.sty can be obtained at:
% http://www.ctan.org/pkg/algorithms
% Also of interest may be the (relatively newer and more customizable)
% algorithmicx.sty package by Szasz Janos:
% http://www.ctan.org/pkg/algorithmicx




% *** ALIGNMENT PACKAGES ***
%
%\usepackage{array}
% Frank Mittelbach's and David Carlisle's array.sty patches and improves
% the standard LaTeX2e array and tabular environments to provide better
% appearance and additional user controls. As the default LaTeX2e table
% generation code is lacking to the point of almost being broken with
% respect to the quality of the end results, all users are strongly
% advised to use an enhanced (at the very least that provided by array.sty)
% set of table tools. array.sty is already installed on most systems. The
% latest version and documentation can be obtained at:
% http://www.ctan.org/pkg/array


% IEEEtran contains the IEEEeqnarray family of commands that can be used to
% generate multiline equations as well as matrices, tables, etc., of high
% quality.




% *** SUBFIGURE PACKAGES ***
%\ifCLASSOPTIONcompsoc
%  \usepackage[caption=false,font=normalsize,labelfont=sf,textfont=sf]{subfig}
%\else
%  \usepackage[caption=false,font=footnotesize]{subfig}
%\fi
% subfig.sty, written by Steven Douglas Cochran, is the modern replacement
% for subfigure.sty, the latter of which is no longer maintained and is
% incompatible with some LaTeX packages including fixltx2e. However,
% subfig.sty requires and automatically loads Axel Sommerfeldt's caption.sty
% which will override IEEEtran.cls' handling of captions and this will result
% in non-IEEE style figure/table captions. To prevent this problem, be sure
% and invoke subfig.sty's "caption=false" package option (available since
% subfig.sty version 1.3, 2005/06/28) as this is will preserve IEEEtran.cls
% handling of captions.
% Note that the Computer Society format requires a larger sans serif font
% than the serif footnote size font used in traditional IEEE formatting
% and thus the need to invoke different subfig.sty package options depending
% on whether compsoc mode has been enabled.
%
% The latest version and documentation of subfig.sty can be obtained at:
% http://www.ctan.org/pkg/subfig




% *** FLOAT PACKAGES ***
%
%\usepackage{fixltx2e}
% fixltx2e, the successor to the earlier fix2col.sty, was written by
% Frank Mittelbach and David Carlisle. This package corrects a few problems
% in the LaTeX2e kernel, the most notable of which is that in current
% LaTeX2e releases, the ordering of single and double column floats is not
% guaranteed to be preserved. Thus, an unpatched LaTeX2e can allow a
% single column figure to be placed prior to an earlier double column
% figure.
% Be aware that LaTeX2e kernels dated 2015 and later have fixltx2e.sty's
% corrections already built into the system in which case a warning will
% be issued if an attempt is made to load fixltx2e.sty as it is no longer
% needed.
% The latest version and documentation can be found at:
% http://www.ctan.org/pkg/fixltx2e


%\usepackage{stfloats}
% stfloats.sty was written by Sigitas Tolusis. This package gives LaTeX2e
% the ability to do double column floats at the bottom of the page as well
% as the top. (e.g., "\begin{figure*}[!b]" is not normally possible in
% LaTeX2e). It also provides a command:
%\fnbelowfloat
% to enable the placement of footnotes below bottom floats (the standard
% LaTeX2e kernel puts them above bottom floats). This is an invasive package
% which rewrites many portions of the LaTeX2e float routines. It may not work
% with other packages that modify the LaTeX2e float routines. The latest
% version and documentation can be obtained at:
% http://www.ctan.org/pkg/stfloats
% Do not use the stfloats baselinefloat ability as the IEEE does not allow
% \baselineskip to stretch. Authors submitting work to the IEEE should note
% that the IEEE rarely uses double column equations and that authors should try
% to avoid such use. Do not be tempted to use the cuted.sty or midfloat.sty
% packages (also by Sigitas Tolusis) as the IEEE does not format its papers in
% such ways.
% Do not attempt to use stfloats with fixltx2e as they are incompatible.
% Instead, use Morten Hogholm'a dblfloatfix which combines the features
% of both fixltx2e and stfloats:
%
% \usepackage{dblfloatfix}
% The latest version can be found at:
% http://www.ctan.org/pkg/dblfloatfix




%\ifCLASSOPTIONcaptionsoff
%  \usepackage[nomarkers]{endfloat}
% \let\MYoriglatexcaption\caption
% \renewcommand{\caption}[2][\relax]{\MYoriglatexcaption[#2]{#2}}
%\fi
% endfloat.sty was written by James Darrell McCauley, Jeff Goldberg and 
% Axel Sommerfeldt. This package may be useful when used in conjunction with 
% IEEEtran.cls'  captionsoff option. Some IEEE journals/societies require that
% submissions have lists of figures/tables at the end of the paper and that
% figures/tables without any captions are placed on a page by themselves at
% the end of the document. If needed, the draftcls IEEEtran class option or
% \CLASSINPUTbaselinestretch interface can be used to increase the line
% spacing as well. Be sure and use the nomarkers option of endfloat to
% prevent endfloat from "marking" where the figures would have been placed
% in the text. The two hack lines of code above are a slight modification of
% that suggested by in the endfloat docs (section 8.4.1) to ensure that
% the full captions always appear in the list of figures/tables - even if
% the user used the short optional argument of \caption[]{}.
% IEEE papers do not typically make use of \caption[]'s optional argument,
% so this should not be an issue. A similar trick can be used to disable
% captions of packages such as subfig.sty that lack options to turn off
% the subcaptions:
% For subfig.sty:
% \let\MYorigsubfloat\subfloat
% \renewcommand{\subfloat}[2][\relax]{\MYorigsubfloat[]{#2}}
% However, the above trick will not work if both optional arguments of
% the \subfloat command are used. Furthermore, there needs to be a
% description of each subfigure *somewhere* and endfloat does not add
% subfigure captions to its list of figures. Thus, the best approach is to
% avoid the use of subfigure captions (many IEEE journals avoid them anyway)
% and instead reference/explain all the subfigures within the main caption.
% The latest version of endfloat.sty and its documentation can obtained at:
% http://www.ctan.org/pkg/endfloat
%
% The IEEEtran \ifCLASSOPTIONcaptionsoff conditional can also be used
% later in the document, say, to conditionally put the References on a 
% page by themselves.




% *** PDF, URL AND HYPERLINK PACKAGES ***
%
%\usepackage{url}
% url.sty was written by Donald Arseneau. It provides better support for
% handling and breaking URLs. url.sty is already installed on most LaTeX
% systems. The latest version and documentation can be obtained at:
% http://www.ctan.org/pkg/url
% Basically, \url{my_url_here}.




% *** Do not adjust lengths that control margins, column widths, etc. ***
% *** Do not use packages that alter fonts (such as pslatex).         ***
% There should be no need to do such things with IEEEtran.cls V1.6 and later.
% (Unless specifically asked to do so by the journal or conference you plan
% to submit to, of course. )


% correct bad hyphenation here
\hyphenation{op-tical net-works semi-conduc-tor}
\usepackage{tikz}
\usetikzlibrary{arrows,shadows}
\usepackage{pgf-umlsd}

\begin{document}
%
% paper title
% Titles are generally capitalized except for words such as a, an, and, as,
% at, but, by, for, in, nor, of, on, or, the, to and up, which are usually
% not capitalized unless they are the first or last word of the title.
% Linebreaks \\ can be used within to get better formatting as desired.
% Do not put math or special symbols in the title.
\title{Design Document for Machine Learning for March Madness}
%
%
% author names and IEEE memberships
% note positions of commas and nonbreaking spaces ( ~ ) LaTeX will not break
% a structure at a ~ so this keeps an author's name from being broken across
% two lines.
% use \thanks{} to gain access to the first footnote area
% a separate \thanks must be used for each paragraph as LaTeX2e's \thanks
% was not built to handle multiple paragraphs
%

\author{Alex~Hoffer,
        Chongxian~Chen,
        and~Jacob~Smith\\Team~Name: Stat~Champs}% <-this % stops a space


% note the % following the last \IEEEmembership and also \thanks - 
% these prevent an unwanted space from occurring between the last author name
% and the end of the author line. i.e., if you had this:
% 
% \author{....lastname \thanks{...} \thanks{...} }
%                     ^------------^------------^----Do not want these spaces!
%
% a space would be appended to the last name and could cause every name on that
% line to be shifted left slightly. This is one of those "LaTeX things". For
% instance, "\textbf{A} \textbf{B}" will typeset as "A B" not "AB". To get
% "AB" then you have to do: "\textbf{A}\textbf{B}"
% \thanks is no different in this regard, so shield the last } of each \thanks
% that ends a line with a % and do not let a space in before the next \thanks.
% Spaces after \IEEEmembership other than the last one are OK (and needed) as
% you are supposed to have spaces between the names. For what it is worth,
% this is a minor point as most people would not even notice if the said evil
% space somehow managed to creep in.



% The paper headers
\markboth{Machine Learning for March Madness Design Document}%
{Shell \MakeLowercase{\textit{et al.}}: Bare Demo of IEEEtran.cls for IEEE Journals}
% The only time the second header will appear is for the odd numbered pages
% after the title page when using the twoside option.
% 
% *** Note that you probably will NOT want to include the author's ***
% *** name in the headers of peer review papers.                   ***
% You can use \ifCLASSOPTIONpeerreview for conditional compilation here if
% you desire.




% If you want to put a publisher's ID mark on the page you can do it like
% this:
%\IEEEpubid{0000--0000/00\$00.00~\copyright~2015 IEEE}
% Remember, if you use this you must call \IEEEpubidadjcol in the second
% column for its text to clear the IEEEpubid mark.



% use for special paper notices
%\IEEEspecialpapernotice{(Invited Paper)}


\renewcommand*\contentsname{Table of Contents}


% make the title area
\maketitle

% As a general rule, do not put math, special symbols or citations
% in the abstract or keywords.


% For peer review papers, you can put extra information on the cover
% page as needed:
% \ifCLASSOPTIONpeerreview
% \begin{center} \bfseries EDICS Category: 3-BBND \end{center}
% \fi
%
% For peerreview papers, this IEEEtran command inserts a page break and
% creates the second title. It will be ignored for other modes.
\IEEEpeerreviewmaketitle

\newpage

\tableofcontents

\newpage


\section{Introduction}
% The very first letter is a 2 line initial drop letter followed
% by the rest of the first word in caps.
% 
% form to use if the first word consists of a single letter:
% \IEEEPARstart{A}{demo} file is ....
% 
% form to use if you need the single drop letter followed by
% normal text (unknown if ever used by the IEEE):
% \IEEEPARstart{A}{}demo file is ....
% 
% Some journals put the first two words in caps:
% \IEEEPARstart{T}{his demo} file is ....
% 
% Here we have the typical use of a "T" for an initial drop letter
% and "HIS" in caps to complete the first word.

\hfill Stat Champs
 
\hfill December 2, 2016
\IEEEPARstart{A}{fter} making reasoned decisions about which technologies to use to implement this module, we must now consider how these technologies will work together. This design document intends to explain how the three technologies each of the Stat Champs have selected to use will complete their assigned task. It will do so through the use of diagrams followed by paragraphs describing what the technology must do and why it must do it. The first three designs were produced by Alex Hoffer. These first three designs are chiefly concerned with the usability of the service, specifically the design of the graphical user interface, the instructions that will be presented to the user, and the display of the machine learning bracket that is generated. In technologies 4 and 5 Jake Smith will talk about how we will obtain and store the sports statistics for easy use by our machine learning algorithms. It is our belief that in order for the tool to be effective in educating students it must be well designed and easy to understand. Chongxian Chen will be responsible for designing and implementing machine learning algorithm. Technologies 6, 7 and 8  will be discussing technology involved in designing machine learning algorithm, namely algorithm library, statistics model and cloud server for hosting the project % You must have at least 2 lines in the paragraph with the drop letter
% (should never be an issue)




\section{Glossary}PyGUI: Graphical user interface API designed specifically for use with the Python programming language. \\
VTK: Data visualization API that will be used with the Python programming language. \\
Client-server architecture: Website format that consists of a user sending and receiving data to a server which also sends and receives data. \\
Webix: JavaScript API that is designed to support graphical user interfaces. \\


\section{Designs}


{\large Design viewpoint 1: Using PyGUI for GUI of webpage}
\\
\\
\begin{center}Design view 1:\end{center}
\begin{sequencediagram}
\newthread[white]{u}{User}
\newinst[3]{p}{PyGUI Formatted Web Page}
\newinst[3]{s}{Submodules}

\begin{call}{u}{Visit web page}{p}{}
\end{call}

\begin{call}{p}{Get statistical categories}{s}{}
\end{call}

\begin{call}{s}{Return statistical categories}{p}{}
\end{call}

\begin{call}{u}{Choose statistical categories}{p}{}
\end{call}

\begin{call}{p}{Send statistical categories}{s}{}
\end{call}

\begin{call}{s}{Return VTK formatted bracket}{p}{}
\end{call}

\begin{call}{p}{Help present VTK formatted bracket}{u}{}
\end{call}
 
\end{sequencediagram}
 
{\large Design Rationale:}
The module needs a web page to operate. This web page needs to be presented in a way that is pleasing to the eye and easy to use, because our intended users are biochemists, not computer scientists. This means the page needs to be as non-esoteric as possible. To achieve this, we will be using PyGUI to format our web page. The precondition of this sequence diagram is that the user has a browser. The postcondition of this diagram are that the user is presented with a bracket that resulted from collaboration between PyGUI and VTK. The flow of events from the perspective of PyGUI is quite simplistic. While a typical client-server architecture will allow us to transfer data back and forth between our submodules, PyGUI must make this data transmission appear as convenient as possible to the user.  
\\
\\
{\large Design viewpoint 2: Using Webix to present instructions for the module}
\\
\\
\begin{center}Design view 2:\end{center}
\begin{sequencediagram}
\newthread[white]{u}{User}
\newinst[3]{p}{PyGUI formatted web page}
\newinst[3]{w}{Webix}
 
\begin{call}{u}{Visit website}{p}{}
\end{call}

\begin{call}{p}{Ask for instructions}{w}{}
\end{call}

\begin{call}{w}{Return instructions}{p}{}
\end{call}

\begin{call}{p}{Present webix formatted instructions}{u}{}
\end{call}

\begin{call}{u}{Click accept}{p}{}
\end{call}

\begin{call}{p}{Present module}{u}{}
\end{call}

\end{sequencediagram}
 
{\large Design Rationale:}
The module needs to present instructions to the user which informs them how to use the service. These instructions should appear seamlessly and should be visually inoffensive. Our main aim is reducing the chance of these instructions appearing unclear and thus maximizing utility. To achieve this, we will be using Webix, which will interact with our client/server code as well as with PyGUI. The precondition of this sequence diagram is that the user has a browser and has loaded the web page. The postconditions are that the user has seen the instructions (hopefully having read them) and has clicked accept to begin the service. The flow of events of the average use case are as follows: 1) the user visits the web page 2) the user is presented with instructions on properly using the module 3) the user reads the instructions 4) the user clicks accept 5) the module is presented. Note that the PyGUI formatted page in this diagram is a bit vague. PyGUI, of course, cannot handle the entire module. The web page will be calling a number of other technologies to present the module, but these technologies are not necessary when only considering how Webix will be used. 
\\
\\
{\large Design viewpoint 3: Generating bracket using VTK}
\\
\\
\begin{center}Design view 3:\end{center}

\begin{sequencediagram}
\newthread[white]{u}{User}
\newinst[3]{py}{PyGUI formatted web page}
\newinst[3]{sci}{Scikit}
\newinst[3]{vtk}{VTK}
 
\begin{call}{u}{Visit site}{py}{}
\end{call}

\begin{call}{u}{Click button}{py}{}
\end{call}

\begin{call}{py}{Provide stat categories from db}{u}{}
\end{call}

\begin{call}{u}{Send chosen stats}{py}{}
\end{call}

\begin{call}{py}{Send chosen stats from db}{sci}{}
\end{call}

\begin{call}{sci}{Send model}{vtk}{}
\end{call}

\begin{call}{vtk}{Send bracket}{py}{}
\end{call}

\begin{call}{py}{Present bracket}{u}{}
\end{call}

\end{sequencediagram}
 
{\large Design Rationale:}
The most essential function of the entire module is presenting a machine learned March Madness bracket. However, the process of machine learning is completely separate from the design of the bracket which will be displayed. Once Scikit generates a machine learned model that reflects the user's chosen statistics, it must communicate to VTK what this data is, and VTK will handle the bracket generation. In order for VTK to produce a bracket for display, a number of preconditions must occur. First, the user must have read and accepted the instructions. Then, the user must have selected data produced by a database. This data must be passed to Scikit, which generates a learned model. The postconditions of this sequence diagram is that the user will have seen the March Madness bracket that results from their chosen statistics. The flow of events for the average use case is as follows: 1) The user reads and accepts the instructions 2) The user selects statistics 3) The server sends these statistics to a machine learning submodule 4) The submodule generates a model 5) The submodule passes the model to VTK 6) VTK transforms the model into a bracket 7) A collaboration of VTK, PyGUI, and the server produce this bracket for the user to view.
\\
\\
{\large Design viewpoint 4: Collection of data for the database}

{\large Design Rationale:}
The Machine Learning algorithms we want to work with get more accurate the more examples and stats you feed into it. Therefore, for the data collection we need to include as many possible data points as possible. We will do this by collecting stats from every major stat website and also pull more advanced data from the hoop-math which breaks down each players moves into more specialized data points. For example, hoop-math with break down a player’s shots from just saying he shot a 2 pointer to telling you where the shot was and how close a defender was to him at that moment. The design to grab the stats is to manually download all the csv files I can from Sports-Reference and upload them to the database then once I am caught up with the current season I will create a python script that takes the stats from all the daily games and uploads it into the database. 

{\large Design viewpoint 5: Database Design}
Design view 4:
Couldn’t get UML database design to post correctly.
{\large Design Rationale:}

For designing the database I have decided to have multiple connected databases. The fist will be a running tally for all the player’s stats, everything from FG percentage to player efficiency and more. The second will be all the team game logs with their opponent’s stats for that game as well. That way we can do look ups for past matchups and analyze how both teams and players did against each other and apply that for future matchups. Then for the final database that will be connected to the game logs will be the advanced game player stats for example it will have play by play stats like who passed to who, when and who shot the ball, from where, and who was guarding him.  The user will be able to choose between either all three databases in there bracket analysis or just two or even one.  Each bracket should be able to bring in stats that should change the outcome of the machine learning predictions.


%%%%%%%%%%%%%Chongxian's design viewpoint%%%%%%%
{\large Design viewpoint 6: SciKit Learn Machine Learning Model}
{\large Design Rationale:}

The project will be implementing the prediction model using python machine learning library SciKit-Learn. More specifically, we will be using the supervised learning regression model. \\
The model will be starting with simple and straightforward input first, and then gradually add more complex and hard-to-predict input. The benefit of starting with straightforward input is that we can easily verify the prediction accuracy of our model. For instance, the most straightforward input could be the match results between two teams in recent years. In most situation the team that wins significantly more in the recent matches should have a higher chance of winning current match. After confirming that our model could predict well on basic inputs, we can then add more complex inputs that may also affect the result of the game like home team status, weather and rest status.\\
After verifying that our model?s prediction accuracy is satisfying, we will be working on enable and disable categories of inputs to meet our project?s education purpose. When enabling all or most categories of inputs, we will be expecting our model to be more accurate. With trivial category of input, our model may show very inaccurate prediction that may even contradict to recent match results. The education purpose of our project will allow users to see a difference when they choose different category of data. \\
When the model is basically complete, we will be testing it with actual matches, particularly the march madness event. We will be continuing modifying our algorithm to enhance the prediction accuracy with live matches. \\
Timeline: Training model with basic inputs: Winter Week 1 to Week 3. Training model with more categories of inputs: Winter Week 4 to Week 6. Enable and disable category of inputs: Winter Week 7 to Week 9. Testing and improving the model with March Madness: Spring Week 2 to Spring Week 5.\\

{\large Design viewpoint 7: Amazon Web Service(AWS) to host the database and computing power}
{\large Design Rationale:}
We will be hosting our database and computing machine on Amazon Web Service. AWS is free for the first 12 months of registration with some limitation of use. After researching on AWS website, we will be able to host one linux instance free with Amazon EC2. We will also be able to hosting a free MySQL database with limited I/O from Amazon RDS. Although the I/O times are limited, but it is large enough for our project(10,000,000 I/Os). Hosting our computing power and database on the cloud is a good idea because we will be able to easily resize our computing power and database with AWS infrastructure when our project grows to a bigger size. \\
To start with AWS free tier for 12 months , we will be creating a group account. First we will be going to EC2 and host a Linux instance. We will have to choose a region among Amazon?s global servers. The one that is most convenient to us will be US West(Oregon). The operating system we will be using will be Ubuntu Server. Ubuntu is very popular among developers and have a large supporting community if we have trouble. Python can run seamlessly on Ubuntu terminal. For the instance type, we have only one option. T2.micro with one core cpu, one Gigabyte Ram and 8 Gigabyte Storage. The Ram is not very big so we may need to be careful with our ram usage or upgrade it in the future. \\
After creating the EC2 instance, we will moving forward to create the RDS database instance. We will be using relational database MySQL to store our data. The free tier also provide us with one core cpu, one Gigabyte Ram and 8 Gigabyte Storage. That should be enough for the teams in NCAA. We can also easily upgrade it when our project grows. When completing the setup of the database on AWS, we need to manually connect it to a database management tools so we can create tables and manage data. A convenient cross-platform tool we can use will be MySQL workbench. It is open-sourced, free and works very well on most platforms including Windows, Mac and Linux. We will be using SSH to connect the database on AWS to MySQL workbench. After that we can manage it like we have learned in class. Running SQL queries or create table using MySQL workbench?s graphical interface. \\
Timeline: Creating EC2 and RDS instance on AWS: Winter week 1 to week 2. Connecting to EC2 and RDS instance: Winter Week 3 to Week 4.\\

{\large Design viewpoint 8: Statistics Model}
{\large Design Rationale:}
With the Statistics Libraries in Python like Py-Statiscs and a large amount of input data in our library, our statistics model is expected to give a good estimate about the weight of each factor in our prediction model. This factor will vary for different teams, but we will have a general statistics equation. After providing it with a large amount of data for each teams, the equation is expected to give respective weight for each team. And then we can supply these weight to our machine learning model to get the final result.\\
First we will be implementing the equation with basic inputs matching results. With only matching results, the equation may doesn?t make too much sense to estimate the weight since there is nothing to compare with. Then we should add a trivial data to the equation so we can tell the difference. We will be expecting the trivial data to have a significantly less weight than recent match results. For the trivial data, the candidates are weather, team colors, etc. We will try the team colors first since that is an easy to collect data category for the beginning of our project. \\
After the basic statistics model makes sense with the data category we provide, we will be adding more category of data like player info, home game status, coach info etc. With a large amount of data, we should be able to a weight of each of these categories. Then we can apply these weights in our prediction model. We will be expecting to see our model become more accurate. With more important categories of data added, our prediction should be more confident and we will be testing it with the march madness results next Spring. With the new data, we will be testing the accuracy of our predictions and improving it. \\
Timeline: Implement statistics model with basic inputs: Winter Week 1 to Week 3. Implement statistics model with more categories of inputs: Winter Week 4 to Week 6. Enable and disable category of inputs and generate perspective weight: Winter Week 7 to Week 9. Testing and improving the statistics model with March Madness: Spring Week 2 to Spring Week 5.\\

% An example of a floating figure using the graphicx package.
% Note that \label must occur AFTER (or within) \caption.
% For figures, \caption should occur after the \includegraphics.
% Note that IEEEtran v1.7 and later has special internal code that
% is designed to preserve the operation of \label within \caption
% even when the captionsoff option is in effect. However, because
% of issues like this, it may be the safest practice to put all your
% \label just after \caption rather than within \caption{}.
%
% Reminder: the "draftcls" or "draftclsnofoot", not "draft", class
% option should be used if it is desired that the figures are to be
% displayed while in draft mode.
%
%\begin{figure}[!t]
%\centering
%\includegraphics[width=2.5in]{myfigure}
% where an .eps filename suffix will be assumed under latex, 
% and a .pdf suffix will be assumed for pdflatex; or what has been declared
% via \DeclareGraphicsExtensions.
%\caption{Simulation results for the network.}
%\label{fig_sim}
%\end{figure}

% Note that the IEEE typically puts floats only at the top, even when this
% results in a large percentage of a column being occupied by floats.


% An example of a double column floating figure using two subfigures.
% (The subfig.sty package must be loaded for this to work.)
% The subfigure \label commands are set within each subfloat command,
% and the \label for the overall figure must come after \caption.
% \hfil is used as a separator to get equal spacing.
% Watch out that the combined width of all the subfigures on a 
% line do not exceed the text width or a line break will occur.
%
%\begin{figure*}[!t]
%\centering
%\subfloat[Case I]{\includegraphics[width=2.5in]{box}%
%\label{fig_first_case}}
%\hfil
%\subfloat[Case II]{\includegraphics[width=2.5in]{box}%
%\label{fig_second_case}}
%\caption{Simulation results for the network.}
%\label{fig_sim}
%\end{figure*}
%
% Note that often IEEE papers with subfigures do not employ subfigure
% captions (using the optional argument to \subfloat[]), but instead will
% reference/describe all of them (a), (b), etc., within the main caption.
% Be aware that for subfig.sty to generate the (a), (b), etc., subfigure
% labels, the optional argument to \subfloat must be present. If a
% subcaption is not desired, just leave its contents blank,
% e.g., \subfloat[].


% An example of a floating table. Note that, for IEEE style tables, the
% \caption command should come BEFORE the table and, given that table
% captions serve much like titles, are usually capitalized except for words
% such as a, an, and, as, at, but, by, for, in, nor, of, on, or, the, to
% and up, which are usually not capitalized unless they are the first or
% last word of the caption. Table text will default to \footnotesize as
% the IEEE normally uses this smaller font for tables.
% The \label must come after \caption as always.
%
%\begin{table}[!t]
%% increase table row spacing, adjust to taste
%\renewcommand{\arraystretch}{1.3}
% if using array.sty, it might be a good idea to tweak the value of
% \extrarowheight as needed to properly center the text within the cells
%\caption{An Example of a Table}
%\label{table_example}
%\centering
%% Some packages, such as MDW tools, offer better commands for making tables
%% than the plain LaTeX2e tabular which is used here.
%\begin{tabular}{|c||c|}
%\hline
%One & Two\\
%\hline
%Three & Four\\
%\hline
%\end{tabular}
%\end{table}


% Note that the IEEE does not put floats in the very first column
% - or typically anywhere on the first page for that matter. Also,
% in-text middle ("here") positioning is typically not used, but it
% is allowed and encouraged for Computer Society conferences (but
% not Computer Society journals). Most IEEE journals/conferences use
% top floats exclusively. 
% Note that, LaTeX2e, unlike IEEE journals/conferences, places
% footnotes above bottom floats. This can be corrected via the
% \fnbelowfloat command of the stfloats package.

\newpage
\begin{section}{Agreement}
\textbf{ }
\vspace{5.0cm}

\noindent\rule{13cm}{0.4pt}\\
Client
\vspace{3.0cm}

\noindent\rule{13cm}{0.4pt}\\
Developer
\vspace{3.0cm}


\noindent\rule{13cm}{0.4pt}\\
Developer
\vspace{3.0cm}


\noindent\rule{13cm}{0.4pt}\\
Developer
\vspace{3.0cm}

\end{section}


% if have a single appendix:
%\appendix[Proof of the Zonklar Equations]
% or
%\appendix  % for no appendix heading
% do not use \section anymore after \appendix, only \section*
% is possibly needed

% use appendices with more than one appendix
% then use \section to start each appendix
% you must declare a \section before using any
% \subsection or using \label (\appendices by itself
% starts a section numbered zero.)
%

% Can use something like this to put references on a page
% by themselves when using endfloat and the captionsoff option.
\ifCLASSOPTIONcaptionsoff
  \newpage
\fi



% trigger a \newpage just before the given reference
% number - used to balance the columns on the last page
% adjust value as needed - may need to be readjusted if
% the document is modified later
%\IEEEtriggeratref{8}
% The "triggered" command can be changed if desired:
%\IEEEtriggercmd{\enlargethispage{-5in}}

% references section

% can use a bibliography generated by BibTeX as a .bbl file
% BibTeX documentation can be easily obtained at:
% http://mirror.ctan.org/biblio/bibtex/contrib/doc/
% The IEEEtran BibTeX style support page is at:
% http://www.michaelshell.org/tex/ieeetran/bibtex/
%\bibliographystyle{IEEEtran}
% argument is your BibTeX string definitions and bibliography database(s)
%\bibliography{IEEEabrv,../bib/paper}
%
% <OR> manually copy in the resultant .bbl file
% set second argument of \begin to the number of references
% (used to reserve space for the reference number labels box)
\newpage
\begin{thebibliography}{1}

\bibitem{Kivy}
  “Kivy: Cross-platform Python Framework for NUI,” \emph{Kivy}. [Online]. Available: https://kivy.org/. [Accessed: 14-Nov-2016].
\bibitem{Dice} 
 D. Bolton, “5 Top Python GUI Frameworks for 2015 - Dice Insights,” \emph{Dice}, Feb-2016. [Online]. Available: http://insights.dice.com/2014/11/26/5-top-python-guis-for-2015/. [Accessed: 14-Nov-2016]. 
\bibitem{Webix}
“JavaScript Framework and HTML5 UI Library for Web App Development-Webix,” \emph{Webix}. [Online]. Available: https://webix.com/. [Accessed: 14-Nov-2016]. 
\bibitem{Best Web Frameworks} 
 UKI,” \emph{Best Web Frameworks}. [Online]. Available: http://www.bestwebframeworks.com/web-framework-review/javascript/117/uki/. [Accessed: 14-Nov-2016]. 
\bibitem{Let's Code Javascript}
J. Shore, “An Unconventional Review of AngularJS,” \emph{Let's Code Javascript}, 14-Jan-2015. [Online]. Available: http://www.letscodejavascript.com/v3/blog/2015/01/angular\_review. [Accessed: 14-Nov-2016]. 
\bibitem{Effbot}
 F. Lungh, “Notes on Tkinter Performance,” \emph{Effbot}, 14-Jul-2002. [Online]. Available: http://effbot.org/zone/tkinter-performance.htm. [Accessed: 14-Nov-2016]. 
 \bibitem{VTK}
 “VTK-Enabled Applications,” \emph{VTK}. [Online]. Available: http://www.vtk.org/. [Accessed: 14-Nov-2016]. 
 \bibitem{Python}
 “Wax GUI Toolkit,” \emph{Python}. [Online]. Available: https://wiki.python.org/moin/wax. [Accessed: 14-Nov-2016]. 
 \bibitem{Amazon Machine Learning}
 “Amazon Machine Learning,” \emph{Amazon}. [Online]. Available: https://aws.amazon.com/machine-learning/. [Accessed: 14-Nov-2016]. 
 \bibitem{SciKit}
 "SciKit Learn,Machine Learning in Python,” \emph{scikit-learn}. [Online]. Available: http://scikit-learn.org/stable/. [Accessed: 14-Nov-2016]. 
 \bibitem{Pylearn2}
 "Pylearn2 devdocumentation,” \emph{Pylearn2}. [Online]. Available: http://deeplearning.net/software/pylearn2/. [Accessed: 14-Nov-2016]. 
 \bibitem{ONID - OSU Network ID}
 "ONID - OSU Network ID,” \emph{Oregon State University}. [Online]. Available: http://onid.oregonstate.edu. [Accessed: 14-Nov-2016]. 
  \bibitem{py-statistics}
 "Python statistics — Mathematical statistics functions,” \emph{Python}. [Online]. Available: https://docs.python.org/3/library/statistics.html. [Accessed: 14-Nov-2016]. 
   \bibitem{scipy.stats}
 "Statistical functions (scipy.stats),” \emph{scipy.stats}. [Online]. Available: https://docs.scipy.org/doc/scipy/reference/stats.html. [Accessed: 14-Nov-2016]. 
 \bibitem{Pandas}
 "Python Data Analysis Library,” \emph{Pandas}. [Online]. Available: http://pandas.pydata.org. [Accessed: 14-Nov-2016]. 

\end{thebibliography}

% biography section
% 
% If you have an EPS/PDF photo (graphicx package needed) extra braces are
% needed around the contents of the optional argument to biography to prevent
% the LaTeX parser from getting confused when it sees the complicated
% \includegraphics command within an optional argument. (You could create
% your own custom macro containing the \includegraphics command to make things
% simpler here.)
%\begin{IEEEbiography}[{\includegraphics[width=1in,height=1.25in,clip,keepaspectratio]{mshell}}]{Michael Shell}
% or if you just want to reserve a space for a photo:


% You can push biographies down or up by placing
% a \vfill before or after them. The appropriate
% use of \vfill depends on what kind of text is
% on the last page and whether or not the columns
% are being equalized.

%\vfill

% Can be used to pull up biographies so that the bottom of the last one
% is flush with the other column.
%\enlargethispage{-5in}

%signature page


% that's all folks
\end{document}
